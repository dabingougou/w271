% Options for packages loaded elsewhere
\PassOptionsToPackage{unicode}{hyperref}
\PassOptionsToPackage{hyphens}{url}
%
\documentclass[
]{article}
\usepackage{amsmath,amssymb}
\usepackage{iftex}
\ifPDFTeX
  \usepackage[T1]{fontenc}
  \usepackage[utf8]{inputenc}
  \usepackage{textcomp} % provide euro and other symbols
\else % if luatex or xetex
  \usepackage{unicode-math} % this also loads fontspec
  \defaultfontfeatures{Scale=MatchLowercase}
  \defaultfontfeatures[\rmfamily]{Ligatures=TeX,Scale=1}
\fi
\usepackage{lmodern}
\ifPDFTeX\else
  % xetex/luatex font selection
\fi
% Use upquote if available, for straight quotes in verbatim environments
\IfFileExists{upquote.sty}{\usepackage{upquote}}{}
\IfFileExists{microtype.sty}{% use microtype if available
  \usepackage[]{microtype}
  \UseMicrotypeSet[protrusion]{basicmath} % disable protrusion for tt fonts
}{}
\makeatletter
\@ifundefined{KOMAClassName}{% if non-KOMA class
  \IfFileExists{parskip.sty}{%
    \usepackage{parskip}
  }{% else
    \setlength{\parindent}{0pt}
    \setlength{\parskip}{6pt plus 2pt minus 1pt}}
}{% if KOMA class
  \KOMAoptions{parskip=half}}
\makeatother
\usepackage{xcolor}
\usepackage[margin=1in]{geometry}
\usepackage{color}
\usepackage{fancyvrb}
\newcommand{\VerbBar}{|}
\newcommand{\VERB}{\Verb[commandchars=\\\{\}]}
\DefineVerbatimEnvironment{Highlighting}{Verbatim}{commandchars=\\\{\}}
% Add ',fontsize=\small' for more characters per line
\usepackage{framed}
\definecolor{shadecolor}{RGB}{248,248,248}
\newenvironment{Shaded}{\begin{snugshade}}{\end{snugshade}}
\newcommand{\AlertTok}[1]{\textcolor[rgb]{0.94,0.16,0.16}{#1}}
\newcommand{\AnnotationTok}[1]{\textcolor[rgb]{0.56,0.35,0.01}{\textbf{\textit{#1}}}}
\newcommand{\AttributeTok}[1]{\textcolor[rgb]{0.13,0.29,0.53}{#1}}
\newcommand{\BaseNTok}[1]{\textcolor[rgb]{0.00,0.00,0.81}{#1}}
\newcommand{\BuiltInTok}[1]{#1}
\newcommand{\CharTok}[1]{\textcolor[rgb]{0.31,0.60,0.02}{#1}}
\newcommand{\CommentTok}[1]{\textcolor[rgb]{0.56,0.35,0.01}{\textit{#1}}}
\newcommand{\CommentVarTok}[1]{\textcolor[rgb]{0.56,0.35,0.01}{\textbf{\textit{#1}}}}
\newcommand{\ConstantTok}[1]{\textcolor[rgb]{0.56,0.35,0.01}{#1}}
\newcommand{\ControlFlowTok}[1]{\textcolor[rgb]{0.13,0.29,0.53}{\textbf{#1}}}
\newcommand{\DataTypeTok}[1]{\textcolor[rgb]{0.13,0.29,0.53}{#1}}
\newcommand{\DecValTok}[1]{\textcolor[rgb]{0.00,0.00,0.81}{#1}}
\newcommand{\DocumentationTok}[1]{\textcolor[rgb]{0.56,0.35,0.01}{\textbf{\textit{#1}}}}
\newcommand{\ErrorTok}[1]{\textcolor[rgb]{0.64,0.00,0.00}{\textbf{#1}}}
\newcommand{\ExtensionTok}[1]{#1}
\newcommand{\FloatTok}[1]{\textcolor[rgb]{0.00,0.00,0.81}{#1}}
\newcommand{\FunctionTok}[1]{\textcolor[rgb]{0.13,0.29,0.53}{\textbf{#1}}}
\newcommand{\ImportTok}[1]{#1}
\newcommand{\InformationTok}[1]{\textcolor[rgb]{0.56,0.35,0.01}{\textbf{\textit{#1}}}}
\newcommand{\KeywordTok}[1]{\textcolor[rgb]{0.13,0.29,0.53}{\textbf{#1}}}
\newcommand{\NormalTok}[1]{#1}
\newcommand{\OperatorTok}[1]{\textcolor[rgb]{0.81,0.36,0.00}{\textbf{#1}}}
\newcommand{\OtherTok}[1]{\textcolor[rgb]{0.56,0.35,0.01}{#1}}
\newcommand{\PreprocessorTok}[1]{\textcolor[rgb]{0.56,0.35,0.01}{\textit{#1}}}
\newcommand{\RegionMarkerTok}[1]{#1}
\newcommand{\SpecialCharTok}[1]{\textcolor[rgb]{0.81,0.36,0.00}{\textbf{#1}}}
\newcommand{\SpecialStringTok}[1]{\textcolor[rgb]{0.31,0.60,0.02}{#1}}
\newcommand{\StringTok}[1]{\textcolor[rgb]{0.31,0.60,0.02}{#1}}
\newcommand{\VariableTok}[1]{\textcolor[rgb]{0.00,0.00,0.00}{#1}}
\newcommand{\VerbatimStringTok}[1]{\textcolor[rgb]{0.31,0.60,0.02}{#1}}
\newcommand{\WarningTok}[1]{\textcolor[rgb]{0.56,0.35,0.01}{\textbf{\textit{#1}}}}
\usepackage{graphicx}
\makeatletter
\newsavebox\pandoc@box
\newcommand*\pandocbounded[1]{% scales image to fit in text height/width
  \sbox\pandoc@box{#1}%
  \Gscale@div\@tempa{\textheight}{\dimexpr\ht\pandoc@box+\dp\pandoc@box\relax}%
  \Gscale@div\@tempb{\linewidth}{\wd\pandoc@box}%
  \ifdim\@tempb\p@<\@tempa\p@\let\@tempa\@tempb\fi% select the smaller of both
  \ifdim\@tempa\p@<\p@\scalebox{\@tempa}{\usebox\pandoc@box}%
  \else\usebox{\pandoc@box}%
  \fi%
}
% Set default figure placement to htbp
\def\fps@figure{htbp}
\makeatother
\setlength{\emergencystretch}{3em} % prevent overfull lines
\providecommand{\tightlist}{%
  \setlength{\itemsep}{0pt}\setlength{\parskip}{0pt}}
\setcounter{secnumdepth}{-\maxdimen} % remove section numbering
\usepackage{bookmark}
\IfFileExists{xurl.sty}{\usepackage{xurl}}{} % add URL line breaks if available
\urlstyle{same}
\hypersetup{
  pdftitle={W271 Assignment 6},
  hidelinks,
  pdfcreator={LaTeX via pandoc}}

\title{W271 Assignment 6}
\author{}
\date{\vspace{-2.5em}}

\begin{document}
\maketitle

\begin{Shaded}
\begin{Highlighting}[]
\FunctionTok{library}\NormalTok{(tidyverse)}
\FunctionTok{library}\NormalTok{(patchwork) }

\FunctionTok{library}\NormalTok{(lubridate)}

\FunctionTok{library}\NormalTok{(tsibble)}
\FunctionTok{library}\NormalTok{(feasts)}
\FunctionTok{library}\NormalTok{(forecast)}

\FunctionTok{library}\NormalTok{(sandwich)}
\FunctionTok{library}\NormalTok{(lmtest)}

\FunctionTok{library}\NormalTok{(nycflights13)}
\FunctionTok{library}\NormalTok{(blsR)}
\end{Highlighting}
\end{Shaded}

\begin{Shaded}
\begin{Highlighting}[]
\FunctionTok{theme\_set}\NormalTok{(}\FunctionTok{theme\_minimal}\NormalTok{())}
\end{Highlighting}
\end{Shaded}

\section{Plot Flights and Weather
Data}\label{plot-flights-and-weather-data}

To start with this homework, you will be using the same data that
Jeffrey uses in the lecture -- US flights data. The data comes from the
packages \texttt{nycflights13}.

\subsection{Question Goals}\label{question-goals}

Our goal with the tasks in this question are to try to familiarize
yourself with some of the key programming concepts related to time
series data -- setting time indexes and key variables, grouping and
indexing on those variables, and producing descriptive plots of data
that is stored in a time series form.

\section{Question 1 - Flights to nice
places}\label{question-1---flights-to-nice-places}

In the package declarations, we have loaded the \texttt{nycflights13}
package. This provides three objects that we are going to use:

\begin{enumerate}
\def\labelenumi{\arabic{enumi}.}
\tightlist
\item
  \texttt{flights};
\item
  \texttt{airports}; and,
\item
  \texttt{weather}.
\end{enumerate}

You can investigate these objects more by issuing a \texttt{?} before
them, to access their documentation.

\newpage

\subsection{(1 point) Create Data}\label{point-create-data}

As stored, both \texttt{flights} and \texttt{weather} are stored a
``plain'' data frames. To begin, cast the \texttt{flights} dataset into
a time series dataset, a \texttt{tsibble}.

\begin{itemize}
\tightlist
\item
  Use the combination of year, month, day, hour, and minute to produce
  the time index. Call this newly mutated variable \texttt{time\_index}.
  There is very good handling of dates inside of the \texttt{lubridate}
  package. There is a nice one-page cheetsheet that Rstudio makes
  available. For this task you might be looking for
  \texttt{lubridate::make\_datetime}.
\item
  Although it may not generally be true, for this work, also assume that
  you can uniquely identify a flight by the carrier and the flight
  number, so you can use these two pieces of information to define the
  \texttt{key}. We need to define a key because in some cases there are
  more than one flight that leave at the same time -- this is because
  the granularity of our time measure is at the minute and it is
  possible for two planes to leave within the same minute.
\end{itemize}

\subsubsection{Answer}\label{answer}

Data is created below. I checked for missing values.

There are more than one way to create the time index, I experimented
with a couple and settled on the method below.

\begin{Shaded}
\begin{Highlighting}[]
\CommentTok{\#flights}
\CommentTok{\#?flights}

\CommentTok{\# Checking missing value}
\ControlFlowTok{if}\NormalTok{ (}\FunctionTok{nrow}\NormalTok{(flights) }\SpecialCharTok{==} \FunctionTok{sum}\NormalTok{(}\SpecialCharTok{!}\FunctionTok{is.na}\NormalTok{(flights}\SpecialCharTok{$}\NormalTok{time\_hour) }\SpecialCharTok{\&} \SpecialCharTok{!}\FunctionTok{is.na}\NormalTok{(flights}\SpecialCharTok{$}\NormalTok{minute))) \{}
  \FunctionTok{print}\NormalTok{(}\StringTok{"No missing values for departure time y{-}m{-}d{-}h{-}m"}\NormalTok{)}
\NormalTok{\}}
\end{Highlighting}
\end{Shaded}

\begin{verbatim}
## [1] "No missing values for departure time y-m-d-h-m"
\end{verbatim}

\begin{Shaded}
\begin{Highlighting}[]
\NormalTok{flights}\SpecialCharTok{$}\NormalTok{time\_index }\OtherTok{\textless{}{-}} 
\NormalTok{  flights}\SpecialCharTok{$}\NormalTok{time\_hour }\SpecialCharTok{+}\NormalTok{ lubridate}\SpecialCharTok{::}\FunctionTok{minutes}\NormalTok{(flights}\SpecialCharTok{$}\NormalTok{minute)}
\end{Highlighting}
\end{Shaded}

\begin{Shaded}
\begin{Highlighting}[]
\NormalTok{flights\_ts }\OtherTok{\textless{}{-}} 
\NormalTok{  tsibble}\SpecialCharTok{::}\FunctionTok{as\_tsibble}\NormalTok{(}
\NormalTok{    flights,}
    \AttributeTok{key =} \FunctionTok{c}\NormalTok{(carrier, flight),}
    \AttributeTok{index =}\NormalTok{ time\_index}
\NormalTok{  )}
\NormalTok{flights\_ts}
\end{Highlighting}
\end{Shaded}

\begin{verbatim}
## # A tsibble: 336,776 x 20 [1m] <America/New_York>
## # Key:       carrier, flight [5,725]
##     year month   day dep_time sched_dep_time dep_delay arr_time sched_arr_time
##    <int> <int> <int>    <int>          <int>     <dbl>    <int>          <int>
##  1  2013    11     3     1531           1540        -9     1653           1725
##  2  2013    11     4     1539           1540        -1     1712           1725
##  3  2013    11     5     1548           1540         8     1708           1725
##  4  2013    11     6     1535           1540        -5     1657           1725
##  5  2013    11     7     1549           1540         9     1733           1725
##  6  2013    11     8     1539           1540        -1     1706           1725
##  7  2013    11     9     1535           1540        -5     1714           1723
##  8  2013    11    10     1538           1540        -2     1718           1725
##  9  2013    11    11     1527           1540       -13     1710           1725
## 10  2013    11    12     1535           1540        -5     1709           1725
## # i 336,766 more rows
## # i 12 more variables: arr_delay <dbl>, carrier <chr>, flight <int>,
## #   tailnum <chr>, origin <chr>, dest <chr>, air_time <dbl>, distance <dbl>,
## #   hour <dbl>, minute <dbl>, time_hour <dttm>, time_index <dttm>
\end{verbatim}

\newpage

\subsection{(1 point) Flights Per Month}\label{point-flights-per-month}

Using \texttt{ggplot}, create a plot of the number of flights per month.
What, if anything, do you note about the total volume of flights
throughout the year? (Don't worry if the plot doesn't tell something
interesting about the data. This data is pretty\ldots{} boring.)

\subsubsection{Answer}\label{answer-1}

\paragraph{Comment:}\label{comment}

The plot shows a relatively stable volume of flights throughout the
year. Total flight volume is lowest in the winter months (January and
February) and then slightly increases and remains high throughout the
summer and fall, demonstrating a minor degree of annual seasonality
common in the travel industry.

\begin{Shaded}
\begin{Highlighting}[]
\NormalTok{flights\_ts }\SpecialCharTok{\%\textgreater{}\%}
  \FunctionTok{index\_by}\NormalTok{(}\AttributeTok{month\_index =} \FunctionTok{yearmonth}\NormalTok{(time\_index)) }\SpecialCharTok{\%\textgreater{}\%}
  \FunctionTok{summarize}\NormalTok{(}\AttributeTok{monthly\_sum =} \FunctionTok{n}\NormalTok{()) }\SpecialCharTok{\%\textgreater{}\%}
  \FunctionTok{ggplot}\NormalTok{(}\FunctionTok{aes}\NormalTok{(}\AttributeTok{x =}\NormalTok{ month\_index, }
             \AttributeTok{y =}\NormalTok{ monthly\_sum)) }\SpecialCharTok{+} 
  \FunctionTok{geom\_line}\NormalTok{() }\SpecialCharTok{+} 
  \FunctionTok{labs}\NormalTok{(}\AttributeTok{title =} \StringTok{"Monthly number of flights departing NYC"}\NormalTok{,}
       \AttributeTok{x =} \StringTok{"Month"}\NormalTok{,}
       \AttributeTok{y =} \StringTok{"Number of Flights"}\NormalTok{)}
\end{Highlighting}
\end{Shaded}

\pandocbounded{\includegraphics[keepaspectratio]{unit_06_files/figure-latex/flights per month-1.pdf}}

\newpage

\subsection{(1 point) The Tropics}\label{point-the-tropics}

Is there a difference in flights to tropical destinations throughout the
year? Use the following concept of a tropical destination:

\begin{quote}
A tropical destination is one who is ``in the tropics'' -- that is, they
are located between the Tropic of Cancer and the Tropic of Capricorn.
\end{quote}

\begin{enumerate}
\def\labelenumi{\arabic{enumi}.}
\tightlist
\item
  Using the \texttt{airports} dataset, create a new variable,
  \texttt{is\_tropical} that notes whether the destination airport is in
  a tropical latitude.
\item
  Join this airports data onto the flights data.
\item
  Produce a plot that shows the volume of flights to tropical and
  non-tropical destinations, counted as a monthly total, throughout the
  year.
\end{enumerate}

\begin{enumerate}
\def\labelenumi{\alph{enumi}.}
\tightlist
\item
  First, try to do this using a \texttt{group\_by} call that groups on
  \texttt{month} and \texttt{is\_tropical}. Why does this not work? What
  is happening when grouping by \texttt{month} while also having a time
  index?
\item
  Instead, you will need to look into \texttt{tsibble::index\_by} and
  combine this with a \texttt{lubridate} ``extractor'' to pull the time
  object that you want out of the \texttt{time\_index} variable that you
  created.
\item
  To produce the plot, \texttt{group\_by(is\_tropical)}, and
  \texttt{index\_by} the month that you extract from your
  \texttt{time\_index}. (This is a bit of a strange part of the
  \texttt{geom\_*} API, but this might be a useful place to use the
  \texttt{geom\_step} geometry to highlight changes in this series.)
\end{enumerate}

\begin{enumerate}
\def\labelenumi{\arabic{enumi}.}
\setcounter{enumi}{3}
\tightlist
\item
  Comment on what you see in the flights to the tropics, compared to
  flight to non-tropical destinations.
\end{enumerate}

\subsubsection{Answer}\label{answer-2}

Using group\_by(month, is\_tropical) with the original flights data
doesn't work because tsibble requires unique time index column
(time\_index) for its output. When we group by a lower-granularity
variable (month) while the time index is of higher granularity (minute),
there is no correct single time index for the resulting summarized rows

\textbf{Comment on the plots} There are airports in the flights data but
not in airports. Looking into the setdiff leads to 4 airports that are
all lcoated in the tropics. Hence I assigned all these flights to
destination tropics.

I created 2 versions because tropical flights are small compared to
others. So the second plot gives a better visual in terms of
seasonality.

Overall, the volume of flights to non-tropical destinations is
overwhelming and shows high stability throughout the year, with some
seasonal variation.

Flights to tropical destinations clearly exhibit annual seasonality,
peaking during the colder months (December/January) when people are
seeking warm weather, and also showing a slight increase during the
summer months.

\begin{Shaded}
\begin{Highlighting}[]
\CommentTok{\#?airports}
\CommentTok{\#airports}

\NormalTok{airports\_augmented }\OtherTok{\textless{}{-}}\NormalTok{ airports }\SpecialCharTok{\%\textgreater{}\%}
  \FunctionTok{mutate}\NormalTok{(}\AttributeTok{is\_tropical =} \FunctionTok{ifelse}\NormalTok{(}
\NormalTok{    lat }\SpecialCharTok{\textgreater{}=} \SpecialCharTok{{-}}\FloatTok{23.5} \SpecialCharTok{\&}\NormalTok{ lat }\SpecialCharTok{\textless{}=} \FloatTok{23.5}\NormalTok{,}
    \StringTok{"tropical"}\NormalTok{, }
    \StringTok{"non{-}tropical"}
\NormalTok{  ))}

\CommentTok{\#unique(airports\_augmented$is\_tropical)}

\NormalTok{flights\_airports\_ts }\OtherTok{\textless{}{-}}
\NormalTok{  flights\_ts }\SpecialCharTok{\%\textgreater{}\%}
  \FunctionTok{left\_join}\NormalTok{(airports\_augmented, }\AttributeTok{by =} \FunctionTok{c}\NormalTok{(}\StringTok{"dest"} \OtherTok{=} \StringTok{"faa"}\NormalTok{)) }\SpecialCharTok{\%\textgreater{}\%}
  
  \CommentTok{\#    Handle the NA values created by missing airport codes (SJU, STT, BQN, PSE).}
  \CommentTok{\#    Since these known missing airports are tropical, we replace NA with "tropical".}
  \FunctionTok{mutate}\NormalTok{(}
    \AttributeTok{is\_tropical =} \FunctionTok{replace\_na}\NormalTok{(is\_tropical, }\StringTok{"tropical"}\NormalTok{)}
\NormalTok{  ) }\SpecialCharTok{\%\textgreater{}\%}
  \CommentTok{\# Convert the final, cleaned column to a factor}
  \FunctionTok{mutate}\NormalTok{(}\AttributeTok{is\_tropical =} \FunctionTok{factor}\NormalTok{(is\_tropical))}
\end{Highlighting}
\end{Shaded}

\begin{Shaded}
\begin{Highlighting}[]
\CommentTok{\#flights  \%\textgreater{}\%}
\CommentTok{\#  group\_by(month, is\_tropical)  \%\textgreater{}\%}
\CommentTok{\#  summarise(total\_flights = n())}
\end{Highlighting}
\end{Shaded}

\begin{Shaded}
\begin{Highlighting}[]
\NormalTok{flights\_airports\_ts }\SpecialCharTok{\%\textgreater{}\%}
  \FunctionTok{group\_by}\NormalTok{(is\_tropical) }\SpecialCharTok{\%\textgreater{}\%}
  \FunctionTok{index\_by}\NormalTok{(}\AttributeTok{month\_index =} \FunctionTok{yearmonth}\NormalTok{(time\_index)) }\SpecialCharTok{\%\textgreater{}\%}
  \FunctionTok{summarize}\NormalTok{(}\AttributeTok{total\_flights =} \FunctionTok{n}\NormalTok{()) }\SpecialCharTok{\%\textgreater{}\%}
  \FunctionTok{ggplot}\NormalTok{(}\FunctionTok{aes}\NormalTok{(}\AttributeTok{x =}\NormalTok{ month\_index, }
             \AttributeTok{y =}\NormalTok{ total\_flights,}
             \AttributeTok{colors =}\NormalTok{ is\_tropical)) }\SpecialCharTok{+} 
  \FunctionTok{geom\_step}\NormalTok{(}\AttributeTok{linewidth =} \DecValTok{1}\NormalTok{) }\SpecialCharTok{+} 
  \FunctionTok{labs}\NormalTok{(}\AttributeTok{title =} \StringTok{"Monthly Flight Volume to Tropical vs. Non{-}Tropical Destinations"}\NormalTok{,}
       \AttributeTok{x =} \StringTok{"Month"}\NormalTok{,}
       \AttributeTok{y =} \StringTok{"Total Flights"}\NormalTok{,}
       \AttributeTok{color =} \StringTok{"Destination Type"}
\NormalTok{  )}
\end{Highlighting}
\end{Shaded}

\pandocbounded{\includegraphics[keepaspectratio]{unit_06_files/figure-latex/successful index by-1.pdf}}

\begin{Shaded}
\begin{Highlighting}[]
\NormalTok{flights\_airports\_ts }\SpecialCharTok{\%\textgreater{}\%}
  \FunctionTok{group\_by}\NormalTok{(is\_tropical) }\SpecialCharTok{\%\textgreater{}\%}
  \FunctionTok{index\_by}\NormalTok{(}\AttributeTok{month\_index =} \FunctionTok{yearmonth}\NormalTok{(time\_index)) }\SpecialCharTok{\%\textgreater{}\%}
  \FunctionTok{summarise}\NormalTok{(}
    \AttributeTok{total\_flights =} \FunctionTok{n}\NormalTok{()) }\SpecialCharTok{\%\textgreater{}\%}
  \CommentTok{\# Because one line is too low,}
  \CommentTok{\# I use another to gain insight}
  \FunctionTok{ggplot}\NormalTok{(}\FunctionTok{aes}\NormalTok{(}\AttributeTok{x =}\NormalTok{ month\_index, }\AttributeTok{y =}\NormalTok{ total\_flights)) }\SpecialCharTok{+}
  \FunctionTok{geom\_step}\NormalTok{(}\AttributeTok{linewidth =} \DecValTok{1}\NormalTok{) }\SpecialCharTok{+}
  
  \CommentTok{\# Add facetting to separate the two series vertically}
  \FunctionTok{facet\_grid}\NormalTok{(is\_tropical }\SpecialCharTok{\textasciitilde{}}\NormalTok{ ., }\AttributeTok{scales =} \StringTok{"free\_y"}\NormalTok{) }\SpecialCharTok{+}
  
  \FunctionTok{labs}\NormalTok{(}
    \AttributeTok{title =} \StringTok{"Monthly Flight Volume by Destination Type"}\NormalTok{,}
    \AttributeTok{x =} \StringTok{"Month"}\NormalTok{,}
    \AttributeTok{y =} \StringTok{"Total Flights"}\NormalTok{,}
    \AttributeTok{color =} \StringTok{"Destination Type"} \CommentTok{\# This will no longer be used}
\NormalTok{  )}
\end{Highlighting}
\end{Shaded}

\pandocbounded{\includegraphics[keepaspectratio]{unit_06_files/figure-latex/successful index by-2.pdf}}

\begin{Shaded}
\begin{Highlighting}[]
\FunctionTok{summary}\NormalTok{(flights\_airports\_ts}\SpecialCharTok{$}\NormalTok{is\_tropical) }
\end{Highlighting}
\end{Shaded}

\begin{verbatim}
## non-tropical     tropical 
##       328467         8309
\end{verbatim}

\begin{Shaded}
\begin{Highlighting}[]
\DocumentationTok{\#\# Temp {-} to be removed}

\CommentTok{\# 1. Get all unique destination codes from flights}
\CommentTok{\#flight\_dests \textless{}{-} unique(flights\_ts$dest)}

\CommentTok{\# 2. Get all unique airport codes from the augmented airports data}
\CommentTok{\#airport\_codes \textless{}{-} unique(airports\_augmented$faa)}

\CommentTok{\# 3. Find which flight destinations are NOT in the airport codes list}
\CommentTok{\#missing\_codes \textless{}{-} setdiff(flight\_dests, airport\_codes)}

\CommentTok{\# Output the missing codes (there are usually a few)}
\CommentTok{\#print(missing\_codes)}
\end{Highlighting}
\end{Shaded}

\newpage

\section{Question 2 - Weather at New York
Airports}\label{question-2---weather-at-new-york-airports}

Our goal in this question is to ask you to re-apply what you know about
producing time series objects to very similarly structurd data.

\subsection{(1 point) Create a time series of
weather}\label{point-create-a-time-series-of-weather}

Turn your attention to the weather data that is provided in the
\texttt{nycflights13::weather} dataset. Produce a \texttt{tsibble} that
uses time as a time index, and \texttt{origin} as a key for this data.
You will notice that there are three origins, ``EWR'', ``JFK'' and
``LGA''.

(Hint: We anticipate that you are going to see the following error on
the first time that you try to convert this data frame:

\begin{verbatim}
Error in `validate_tsibble()`:
A valid tsibble must have distinct rows identified by key and index.
Please use `duplicates()` to check the duplicated rows.
Run `rlang::last_error()` to see where the error occurred.
\end{verbatim}

This is a \emph{very} helpful error, with a helpful error message. If
you see this error message, we suggest doing as the message suggests,
and look into the \texttt{duplicates()} function to determine what the
issue is. Once you have found the issue, (1) document the issue; (2)
propose a solution that seems reasonable; and, (3) implement your
proposed solution and keep it moving to answer this question.

\subsubsection{Answer}\label{answer-3}

\textbf{Anticipated Issue:}

The base nycflights13::weather data is said to contain duplicates. But I
didn't run into the error message, nor did the code

\begin{quote}
weather \%\textgreater\% duplicates(key = origin, index = time\_hour)
\end{quote}

produced non-zero lines. So the data looks okay.

\begin{Shaded}
\begin{Highlighting}[]
\CommentTok{\# weather \%\textgreater{}\%}
\CommentTok{\#   mutate(time\_index = make\_datetime(year, month, day, hour)) \%\textgreater{}\%}
\CommentTok{\#   duplicates(key = origin, index = time\_index) \%\textgreater{}\%}
\CommentTok{\#   glimpse()}
\end{Highlighting}
\end{Shaded}

\begin{Shaded}
\begin{Highlighting}[]
\CommentTok{\#weather}
\CommentTok{\#?weather}

\CommentTok{\# Okay if producing 0 row}
\NormalTok{weather }\SpecialCharTok{\%\textgreater{}\%}
  \FunctionTok{duplicates}\NormalTok{(}\AttributeTok{key =}\NormalTok{ origin, }\AttributeTok{index =}\NormalTok{ time\_hour)}
\end{Highlighting}
\end{Shaded}

\begin{verbatim}
## # A tibble: 0 x 15
## # i 15 variables: origin <chr>, year <int>, month <int>, day <int>, hour <int>,
## #   temp <dbl>, dewp <dbl>, humid <dbl>, wind_dir <dbl>, wind_speed <dbl>,
## #   wind_gust <dbl>, precip <dbl>, pressure <dbl>, visib <dbl>,
## #   time_hour <dttm>
\end{verbatim}

\begin{Shaded}
\begin{Highlighting}[]
\NormalTok{weather\_ts }\OtherTok{\textless{}{-}} 
  \FunctionTok{as\_tsibble}\NormalTok{(}
\NormalTok{    weather,}
    \AttributeTok{key =}\NormalTok{ origin,}
    \AttributeTok{index =}\NormalTok{ time\_hour}
\NormalTok{  )}

\NormalTok{weather\_ts}
\end{Highlighting}
\end{Shaded}

\begin{verbatim}
## # A tsibble: 26,115 x 15 [1h] <America/New_York>
## # Key:       origin [3]
##    origin  year month   day  hour  temp  dewp humid wind_dir wind_speed
##    <chr>  <int> <int> <int> <int> <dbl> <dbl> <dbl>    <dbl>      <dbl>
##  1 EWR     2013     1     1     1  39.0  26.1  59.4      270      10.4 
##  2 EWR     2013     1     1     2  39.0  27.0  61.6      250       8.06
##  3 EWR     2013     1     1     3  39.0  28.0  64.4      240      11.5 
##  4 EWR     2013     1     1     4  39.9  28.0  62.2      250      12.7 
##  5 EWR     2013     1     1     5  39.0  28.0  64.4      260      12.7 
##  6 EWR     2013     1     1     6  37.9  28.0  67.2      240      11.5 
##  7 EWR     2013     1     1     7  39.0  28.0  64.4      240      15.0 
##  8 EWR     2013     1     1     8  39.9  28.0  62.2      250      10.4 
##  9 EWR     2013     1     1     9  39.9  28.0  62.2      260      15.0 
## 10 EWR     2013     1     1    10  41    28.0  59.6      260      13.8 
## # i 26,105 more rows
## # i 5 more variables: wind_gust <dbl>, precip <dbl>, pressure <dbl>,
## #   visib <dbl>, time_hour <dttm>
\end{verbatim}

\begin{Shaded}
\begin{Highlighting}[]
\NormalTok{weather\_ts }\OtherTok{\textless{}{-}} 
  \FunctionTok{as\_tsibble}\NormalTok{(}
\NormalTok{    weather,}
    \AttributeTok{key =}\NormalTok{ origin,}
    \AttributeTok{index =}\NormalTok{ time\_hour}
\NormalTok{  )}

\NormalTok{weather\_ts}
\end{Highlighting}
\end{Shaded}

\begin{verbatim}
## # A tsibble: 26,115 x 15 [1h] <America/New_York>
## # Key:       origin [3]
##    origin  year month   day  hour  temp  dewp humid wind_dir wind_speed
##    <chr>  <int> <int> <int> <int> <dbl> <dbl> <dbl>    <dbl>      <dbl>
##  1 EWR     2013     1     1     1  39.0  26.1  59.4      270      10.4 
##  2 EWR     2013     1     1     2  39.0  27.0  61.6      250       8.06
##  3 EWR     2013     1     1     3  39.0  28.0  64.4      240      11.5 
##  4 EWR     2013     1     1     4  39.9  28.0  62.2      250      12.7 
##  5 EWR     2013     1     1     5  39.0  28.0  64.4      260      12.7 
##  6 EWR     2013     1     1     6  37.9  28.0  67.2      240      11.5 
##  7 EWR     2013     1     1     7  39.0  28.0  64.4      240      15.0 
##  8 EWR     2013     1     1     8  39.9  28.0  62.2      250      10.4 
##  9 EWR     2013     1     1     9  39.9  28.0  62.2      260      15.0 
## 10 EWR     2013     1     1    10  41    28.0  59.6      260      13.8 
## # i 26,105 more rows
## # i 5 more variables: wind_gust <dbl>, precip <dbl>, pressure <dbl>,
## #   visib <dbl>, time_hour <dttm>
\end{verbatim}

\newline

\subsection{(4 points) Plot temperature}\label{points-plot-temperature}

With this weather data, produce the following figure of the temperature
every hour, for each of the origins.

This figure contains five separate plots:

\begin{itemize}
\tightlist
\item
  One that shows the entire year's temperature data;
\item
  Two that show the month of January and July; and,
\item
  Two that show the first week of January and July.
\end{itemize}

You might think of these plots as ``zooming in'' on the time series to
show more detail.

In your workflow, first create each of the plots. Then, use the
\texttt{patchwork} package to compose each of these plots into a single
figure.

After you produce this figure, comment on what you notice at each of
these scales and the figure overall.

\subsubsection{Answer}\label{answer-4}

\textbf{Commentary}

Entire Year: The entire series is dominated by strong annual
seasonality, showing a large, smooth sinusoidal pattern. All three
airport temperatures track closely, indicating small temperature
differences across the NYC metropolitan area.

Monthly (January and July): At this scale, the daily cycle becomes
apparent. January shows low but fluctuating temperatures. July shows
consistently warm temperatures, clearly illustrating the seasonal
contrast.

Weekly/Daily: The daily cycle is dominant. Temperature rises during the
day and falls at night in a smooth, predictable manner. High-frequency
noise and short-term volatility are visible at this finest scale.

\begin{Shaded}
\begin{Highlighting}[]
\NormalTok{yearly\_plot }\OtherTok{\textless{}{-}} 
\NormalTok{  weather\_ts }\SpecialCharTok{\%\textgreater{}\%}
  \FunctionTok{ggplot}\NormalTok{(}\FunctionTok{aes}\NormalTok{(}
    \AttributeTok{x =}\NormalTok{ time\_hour,}
    \AttributeTok{y =}\NormalTok{ temp,}
    \AttributeTok{color =}\NormalTok{ origin}
\NormalTok{  )) }\SpecialCharTok{+}
  \FunctionTok{geom\_line}\NormalTok{(}\AttributeTok{linewidth =} \FloatTok{0.25}\NormalTok{) }\SpecialCharTok{+} 
  \FunctionTok{labs}\NormalTok{(}
    \AttributeTok{title =} \StringTok{"Full Year Weather Data"}\NormalTok{,}
    \AttributeTok{x =} \StringTok{"Time"}\NormalTok{,}
    \AttributeTok{y =} \StringTok{"Tempreture Farenheit"}
\NormalTok{  )}

\NormalTok{yearly\_plot}
\end{Highlighting}
\end{Shaded}

\pandocbounded{\includegraphics[keepaspectratio]{unit_06_files/figure-latex/plot yearly temperature-1.pdf}}

\begin{Shaded}
\begin{Highlighting}[]
\NormalTok{january\_plot }\OtherTok{\textless{}{-}} 
\NormalTok{  weather\_ts }\SpecialCharTok{\%\textgreater{}\%}
  \FunctionTok{filter}\NormalTok{(}\FunctionTok{month}\NormalTok{(time\_hour) }\SpecialCharTok{==} \DecValTok{1}\NormalTok{) }\SpecialCharTok{\%\textgreater{}\%}
  \FunctionTok{ggplot}\NormalTok{(}\FunctionTok{aes}\NormalTok{(}
    \AttributeTok{x =}\NormalTok{ time\_hour,}
    \AttributeTok{y =}\NormalTok{ temp,}
    \AttributeTok{color =}\NormalTok{ origin}
\NormalTok{  )) }\SpecialCharTok{+} 
  \FunctionTok{geom\_line}\NormalTok{(}\AttributeTok{linewidth =} \FloatTok{0.2}\NormalTok{) }\SpecialCharTok{+} 
  \FunctionTok{labs}\NormalTok{(}
    \AttributeTok{title =} \StringTok{"2013 January Tempreture"}\NormalTok{,}
    \AttributeTok{x =} \StringTok{"Time {-} Hour"}\NormalTok{,}
    \AttributeTok{y =} \StringTok{"Tempreture in Farenheit"}
\NormalTok{  )}

\NormalTok{january\_plot}
\end{Highlighting}
\end{Shaded}

\pandocbounded{\includegraphics[keepaspectratio]{unit_06_files/figure-latex/plot january and july temperature-1.pdf}}

\begin{Shaded}
\begin{Highlighting}[]
\NormalTok{july\_plot }\OtherTok{\textless{}{-}}   
\NormalTok{  weather\_ts }\SpecialCharTok{\%\textgreater{}\%}
  \FunctionTok{filter}\NormalTok{(}\FunctionTok{month}\NormalTok{(time\_hour) }\SpecialCharTok{==} \DecValTok{7}\NormalTok{) }\SpecialCharTok{\%\textgreater{}\%}
  \FunctionTok{ggplot}\NormalTok{(}\FunctionTok{aes}\NormalTok{(}
    \AttributeTok{x =}\NormalTok{ time\_hour,}
    \AttributeTok{y =}\NormalTok{ temp,}
    \AttributeTok{color =}\NormalTok{ origin}
\NormalTok{  )) }\SpecialCharTok{+} 
  \FunctionTok{geom\_line}\NormalTok{(}\AttributeTok{linewidth =} \FloatTok{0.2}\NormalTok{) }\SpecialCharTok{+} 
  \FunctionTok{labs}\NormalTok{(}
    \AttributeTok{title =} \StringTok{"2013 July Tempreture"}\NormalTok{,}
    \AttributeTok{x =} \StringTok{"Time {-} Hour"}\NormalTok{,}
    \AttributeTok{y =} \StringTok{"Tempreture in Farenheit"}
\NormalTok{  )}

\NormalTok{july\_plot}
\end{Highlighting}
\end{Shaded}

\pandocbounded{\includegraphics[keepaspectratio]{unit_06_files/figure-latex/plot january and july temperature-2.pdf}}

\begin{Shaded}
\begin{Highlighting}[]
\NormalTok{january\_first\_week }\OtherTok{\textless{}{-}}
\NormalTok{  weather\_ts }\SpecialCharTok{\%\textgreater{}\%}
  \FunctionTok{filter}\NormalTok{((}\FunctionTok{month}\NormalTok{(time\_hour) }\SpecialCharTok{==} \DecValTok{1}\NormalTok{) }\SpecialCharTok{\&}\NormalTok{ (}\FunctionTok{day}\NormalTok{(time\_hour) }\SpecialCharTok{\textless{}=} \DecValTok{7}\NormalTok{)) }\SpecialCharTok{\%\textgreater{}\%}
  \FunctionTok{ggplot}\NormalTok{(}\FunctionTok{aes}\NormalTok{(}
    \AttributeTok{x =}\NormalTok{ time\_hour,}
    \AttributeTok{y =}\NormalTok{ temp,}
    \AttributeTok{color =}\NormalTok{ origin}
\NormalTok{  )) }\SpecialCharTok{+} 
  \FunctionTok{geom\_line}\NormalTok{(}\AttributeTok{linewidth =} \FloatTok{0.5}\NormalTok{) }\SpecialCharTok{+} 
  \FunctionTok{labs}\NormalTok{(}
    \AttributeTok{title =} \StringTok{"2013 January First Week Tempreture"}\NormalTok{,}
    \AttributeTok{x =} \StringTok{"Time {-} Hour"}\NormalTok{,}
    \AttributeTok{y =} \StringTok{"Tempreture in Farenheit"}
\NormalTok{  )}

\NormalTok{january\_first\_week}
\end{Highlighting}
\end{Shaded}

\pandocbounded{\includegraphics[keepaspectratio]{unit_06_files/figure-latex/plot weekly temperatures-1.pdf}}

\begin{Shaded}
\begin{Highlighting}[]
\NormalTok{july\_first\_week }\OtherTok{\textless{}{-}}
\NormalTok{  weather\_ts }\SpecialCharTok{\%\textgreater{}\%}
  \FunctionTok{filter}\NormalTok{((}\FunctionTok{month}\NormalTok{(time\_hour) }\SpecialCharTok{==} \DecValTok{7}\NormalTok{) }\SpecialCharTok{\&}\NormalTok{ (}\FunctionTok{day}\NormalTok{(time\_hour) }\SpecialCharTok{\textless{}=} \DecValTok{7}\NormalTok{)) }\SpecialCharTok{\%\textgreater{}\%}
  \FunctionTok{ggplot}\NormalTok{(}\FunctionTok{aes}\NormalTok{(}
    \AttributeTok{x =}\NormalTok{ time\_hour,}
    \AttributeTok{y =}\NormalTok{ temp,}
    \AttributeTok{color =}\NormalTok{ origin}
\NormalTok{  )) }\SpecialCharTok{+} 
  \FunctionTok{geom\_line}\NormalTok{(}\AttributeTok{linewidth =} \FloatTok{0.5}\NormalTok{) }\SpecialCharTok{+} 
  \FunctionTok{labs}\NormalTok{(}
    \AttributeTok{title =} \StringTok{"2013 July First Week Tempreture"}\NormalTok{,}
    \AttributeTok{x =} \StringTok{"Time {-} Hour"}\NormalTok{,}
    \AttributeTok{y =} \StringTok{"Tempreture in Farenheit"}
\NormalTok{  )}

\NormalTok{july\_first\_week}
\end{Highlighting}
\end{Shaded}

\pandocbounded{\includegraphics[keepaspectratio]{unit_06_files/figure-latex/plot weekly temperatures-2.pdf}}

\begin{Shaded}
\begin{Highlighting}[]
\NormalTok{patch\_fig }\OtherTok{\textless{}{-}}
\NormalTok{yearly\_plot }\SpecialCharTok{/}
\NormalTok{  (january\_plot }\SpecialCharTok{|}\NormalTok{ july\_plot) }\SpecialCharTok{/}
\NormalTok{  (january\_first\_week }\SpecialCharTok{|}\NormalTok{ july\_first\_week) }\SpecialCharTok{+}
  \FunctionTok{plot\_annotation}\NormalTok{(}
    \AttributeTok{title    =} \StringTok{\textquotesingle{}Temperature at New York City Airports\textquotesingle{}}\NormalTok{,}
    \AttributeTok{subtitle =} \StringTok{\textquotesingle{}Many Different Views\textquotesingle{}}\NormalTok{,}
    \AttributeTok{tag\_levels =} \StringTok{\textquotesingle{}A\textquotesingle{}}\NormalTok{) }\SpecialCharTok{\&} 
  \FunctionTok{labs}\NormalTok{(}\AttributeTok{x =} \ConstantTok{NULL}\NormalTok{, }\AttributeTok{y =} \StringTok{\textquotesingle{}Temperature\textquotesingle{}}\NormalTok{) }


\NormalTok{patch\_fig}
\end{Highlighting}
\end{Shaded}

\pandocbounded{\includegraphics[keepaspectratio]{unit_06_files/figure-latex/compose five plots using patchwork-1.pdf}}

\newpage

\subsection{(1 point) Hourly ACF}\label{point-hourly-acf}

At the hourly level, produce an ACF and a lag plot at JFK. What do you
learn from these plots? (Note that you can suppress all the coloring in
the \texttt{gg\_lag} call if you pass an additional argument,
\texttt{color\ =\ 1}.)

\subsubsection{Answer}\label{answer-5}

\textbf{Commentary}

ACF Plot: The ACF coefficients show a slow decay with a clear wave-like
pattern peaking every 24 lags. This indicates the series is highly
persistent ( hence non-stationary) and possesses a strong daily
seasonality.

Lag Plot: The points form a tight, upward-sloping cluster. This confirms
visually the high positive correlation---the temperature today at a
given hour is a reliable predictor of the temperature 24 hours ago.

\begin{Shaded}
\begin{Highlighting}[]
\NormalTok{hourly\_acf }\OtherTok{\textless{}{-}} 
\NormalTok{  weather\_ts }\SpecialCharTok{\%\textgreater{}\%}
    \FunctionTok{filter}\NormalTok{(origin }\SpecialCharTok{==} \StringTok{"JFK"}\NormalTok{) }\SpecialCharTok{\%\textgreater{}\%}
    \FunctionTok{fill\_gaps}\NormalTok{() }\SpecialCharTok{\%\textgreater{}\%}
    \FunctionTok{ACF}\NormalTok{(temp, }\AttributeTok{lag\_max =} \DecValTok{48}\NormalTok{) }\SpecialCharTok{\%\textgreater{}\%}
    \FunctionTok{autoplot}\NormalTok{() }\SpecialCharTok{+}
    \FunctionTok{labs}\NormalTok{(}
      \AttributeTok{title =} \StringTok{"JFK Hourly Tempreture ACF"}
\NormalTok{    )}
  
\NormalTok{hourly\_acf}
\end{Highlighting}
\end{Shaded}

\pandocbounded{\includegraphics[keepaspectratio]{unit_06_files/figure-latex/hourly plots-1.pdf}}

\begin{Shaded}
\begin{Highlighting}[]
\NormalTok{hourly\_lag }\OtherTok{\textless{}{-}} 
\NormalTok{  weather\_ts }\SpecialCharTok{\%\textgreater{}\%}
  \FunctionTok{filter}\NormalTok{(origin }\SpecialCharTok{==} \StringTok{"JFK"}\NormalTok{) }\SpecialCharTok{\%\textgreater{}\%}
  \FunctionTok{gg\_lag}\NormalTok{(temp, }\AttributeTok{.log =} \DecValTok{9}\NormalTok{, }\AttributeTok{color =} \DecValTok{1}\NormalTok{) }\SpecialCharTok{+} 
  \FunctionTok{labs}\NormalTok{(}\AttributeTok{title =} \StringTok{"JFK Hourly Tempreture Lag Plot"}\NormalTok{)}
\end{Highlighting}
\end{Shaded}

\begin{verbatim}
## Warning: `gg_lag()` was deprecated in feasts 0.4.2.
## i Please use `ggtime::gg_lag()` instead.
## This warning is displayed once every 8 hours.
## Call `lifecycle::last_lifecycle_warnings()` to see where this warning was
## generated.
\end{verbatim}

\begin{verbatim}
## Warning in lag_geom(..., arrow = arrow): Ignoring unknown parameters: `.log`
\end{verbatim}

\begin{Shaded}
\begin{Highlighting}[]
\NormalTok{hourly\_lag}
\end{Highlighting}
\end{Shaded}

\pandocbounded{\includegraphics[keepaspectratio]{unit_06_files/figure-latex/hourly plots-2.pdf}}

\begin{Shaded}
\begin{Highlighting}[]
\CommentTok{\# Combine and display the plots}
\NormalTok{hourly\_acf }\SpecialCharTok{|}\NormalTok{ hourly\_lag}
\end{Highlighting}
\end{Shaded}

\pandocbounded{\includegraphics[keepaspectratio]{unit_06_files/figure-latex/hourly plots-3.pdf}}

\begin{Shaded}
\begin{Highlighting}[]
\CommentTok{\#hourly\_acf | hourly\_lag}
\end{Highlighting}
\end{Shaded}

\newpage

\subsection{(1 point) Weekly ACF}\label{point-weekly-acf}

At the weekly level, produce an ACF and a lag plot of the weekly average
temperature at JFK. What do you learn from these plots?

\subsubsection{Answer}\label{answer-6}

\textbf{Comment} The aggregation to weekly average removes the daily
cycle seen in the hourly data. The ACF still decays slowly, indicating
the series being non-stationary, but the most important feature is the
large, periodic peak around Lag 52 (one year). This clearly highlights
the annual seasonality of the average temperature.

\begin{Shaded}
\begin{Highlighting}[]
\NormalTok{weekly\_acf }\OtherTok{\textless{}{-}} 
\NormalTok{  weather\_ts }\SpecialCharTok{\%\textgreater{}\%}
\CommentTok{\#  fill\_gaps() \%\textgreater{}\%}
  \FunctionTok{filter}\NormalTok{(origin }\SpecialCharTok{==} \StringTok{"JFK"}\NormalTok{) }\SpecialCharTok{\%\textgreater{}\%}
  \FunctionTok{index\_by}\NormalTok{(}\AttributeTok{week\_index =} \FunctionTok{yearweek}\NormalTok{(time\_hour)) }\SpecialCharTok{\%\textgreater{}\%}
  \FunctionTok{summarize}\NormalTok{(}
    \AttributeTok{avg\_temp =} \FunctionTok{mean}\NormalTok{(temp, }\AttributeTok{na.rm =} \ConstantTok{TRUE}\NormalTok{),}
    \AttributeTok{.groups =} \StringTok{"drop"}
\NormalTok{  ) }\SpecialCharTok{\%\textgreater{}\%}
  \FunctionTok{ACF}\NormalTok{(avg\_temp, }\AttributeTok{lag\_max =} \DecValTok{52}\NormalTok{) }\SpecialCharTok{\%\textgreater{}\%}
  \FunctionTok{autoplot}\NormalTok{() }\SpecialCharTok{+} 
  \FunctionTok{labs}\NormalTok{(}\AttributeTok{title =} \StringTok{"JFK Weekly Average Temperature ACF"}\NormalTok{)}

\NormalTok{weekly\_acf}
\end{Highlighting}
\end{Shaded}

\pandocbounded{\includegraphics[keepaspectratio]{unit_06_files/figure-latex/weekly average ACF plot-1.pdf}}

\begin{Shaded}
\begin{Highlighting}[]
\NormalTok{weekly\_lag }\OtherTok{\textless{}{-}} 
\NormalTok{  weather\_ts }\SpecialCharTok{\%\textgreater{}\%}
\CommentTok{\#  fill\_gaps() \%\textgreater{}\%}
  \FunctionTok{filter}\NormalTok{(origin }\SpecialCharTok{==} \StringTok{"JFK"}\NormalTok{) }\SpecialCharTok{\%\textgreater{}\%}
  \FunctionTok{index\_by}\NormalTok{(}\AttributeTok{week\_index =} \FunctionTok{yearweek}\NormalTok{(time\_hour)) }\SpecialCharTok{\%\textgreater{}\%}
  \FunctionTok{summarize}\NormalTok{(}
    \AttributeTok{avg\_temp =} \FunctionTok{mean}\NormalTok{(temp, }\AttributeTok{na.rm =} \ConstantTok{TRUE}\NormalTok{),}
    \AttributeTok{.groups =} \StringTok{"drop"}
\NormalTok{  ) }\SpecialCharTok{\%\textgreater{}\%}
  \FunctionTok{gg\_lag}\NormalTok{(avg\_temp, }\AttributeTok{.lag =} \DecValTok{9}\NormalTok{, }\AttributeTok{color =} \DecValTok{1}\NormalTok{) }\SpecialCharTok{+} 
  \FunctionTok{labs}\NormalTok{(}\AttributeTok{title =} \StringTok{"JFK Weekly Average Temperature Lag Plot"}\NormalTok{)}
\end{Highlighting}
\end{Shaded}

\begin{verbatim}
## Warning in lag_geom(..., arrow = arrow): Ignoring unknown parameters: `.lag`
\end{verbatim}

\begin{Shaded}
\begin{Highlighting}[]
\NormalTok{weekly\_lag}
\end{Highlighting}
\end{Shaded}

\pandocbounded{\includegraphics[keepaspectratio]{unit_06_files/figure-latex/weekly average ACF plot-2.pdf}}

\begin{Shaded}
\begin{Highlighting}[]
\DocumentationTok{\#\# make the plot}
\end{Highlighting}
\end{Shaded}

\newpage

\subsection{(1 point) Monthly ACF}\label{point-monthly-acf}

At the monthly level, produce an ACF plot of the monthly average
temperature at JFK. What do you learn from these plots?

\subsubsection{Answer}\label{answer-7}

\textbf{Comment:} The monthly ACF shows decay and an annual seasonality
(even with one year worth of data). This again points to
non-stationarity.

\begin{Shaded}
\begin{Highlighting}[]
\NormalTok{monthly\_acf }\OtherTok{\textless{}{-}} 
\NormalTok{  weather\_ts }\SpecialCharTok{\%\textgreater{}\%}
\CommentTok{\#  fill\_gaps() \%\textgreater{}\%}
  \FunctionTok{filter}\NormalTok{(origin }\SpecialCharTok{==} \StringTok{"JFK"}\NormalTok{) }\SpecialCharTok{\%\textgreater{}\%}
  \FunctionTok{index\_by}\NormalTok{(}\AttributeTok{month\_index =} \FunctionTok{yearmonth}\NormalTok{(time\_hour)) }\SpecialCharTok{\%\textgreater{}\%}
  \FunctionTok{summarize}\NormalTok{(}
    \AttributeTok{avg\_temp =} \FunctionTok{mean}\NormalTok{(temp, }\AttributeTok{rm.na =} \ConstantTok{TRUE}\NormalTok{),}
    \AttributeTok{.group =} \StringTok{"drop"}\NormalTok{) }\SpecialCharTok{\%\textgreater{}\%}
  \FunctionTok{ACF}\NormalTok{(avg\_temp) }\SpecialCharTok{\%\textgreater{}\%}
  \FunctionTok{autoplot}\NormalTok{() }\SpecialCharTok{+} 
  \FunctionTok{labs}\NormalTok{(}
    \AttributeTok{title =} \StringTok{"JFK 2013 Monthly Avg. Temp. ACF"}
\NormalTok{  )}
  
\NormalTok{monthly\_acf}
\end{Highlighting}
\end{Shaded}

\pandocbounded{\includegraphics[keepaspectratio]{unit_06_files/figure-latex/monthly average ACF plot-1.pdf}}

\begin{Shaded}
\begin{Highlighting}[]
\NormalTok{monthly\_lag }\OtherTok{\textless{}{-}} 
\NormalTok{  weather\_ts }\SpecialCharTok{\%\textgreater{}\%}
  \FunctionTok{filter}\NormalTok{(origin }\SpecialCharTok{==} \StringTok{"JFK"}\NormalTok{) }\SpecialCharTok{\%\textgreater{}\%}
  \FunctionTok{index\_by}\NormalTok{(}\AttributeTok{month\_index =} \FunctionTok{yearmonth}\NormalTok{(time\_hour)) }\SpecialCharTok{\%\textgreater{}\%}
  \FunctionTok{summarize}\NormalTok{(}
    \AttributeTok{avg\_temp =} \FunctionTok{mean}\NormalTok{(temp, }\AttributeTok{na.rm =} \ConstantTok{TRUE}\NormalTok{),}
    \AttributeTok{.group =} \StringTok{"drop"}
\NormalTok{  ) }\SpecialCharTok{\%\textgreater{}\%}
  \FunctionTok{gg\_lag}\NormalTok{(avg\_temp, }\AttributeTok{.lag =} \DecValTok{9}\NormalTok{, }\AttributeTok{color =} \DecValTok{1}\NormalTok{) }\SpecialCharTok{+} 
  \FunctionTok{labs}\NormalTok{(}\AttributeTok{title =} \StringTok{"JFK 2013 Monthly "}\NormalTok{)}
\end{Highlighting}
\end{Shaded}

\begin{verbatim}
## Warning in lag_geom(..., arrow = arrow): Ignoring unknown parameters: `.lag`
\end{verbatim}

\begin{Shaded}
\begin{Highlighting}[]
\NormalTok{monthly\_lag}
\end{Highlighting}
\end{Shaded}

\pandocbounded{\includegraphics[keepaspectratio]{unit_06_files/figure-latex/monthly average ACF plot-2.pdf}}

\begin{Shaded}
\begin{Highlighting}[]
\DocumentationTok{\#\# make the plot}
\end{Highlighting}
\end{Shaded}

\newpage

\section{Question 3 - BLS Data}\label{question-3---bls-data}

This is the last exercise for this assignment. Here, we're going to do
the same work that you have done twice before, but against ``live'' data
that comes from the United States' Bureau of Labor Statistics.

Recall that in the lecture, Jeffrey identifies the unemployment rate as
an example of a time series. You can get to this data from the public
web-site. To do so, head here:

\begin{itemize}
\tightlist
\item
  www.bls.gov \textgreater{} Data Tools \textgreater{} BLS Popular
  Series
\item
  Then check the box for
  \texttt{Unemployment\ Rate\ (Seasonally\ Adjusted)} and
  \texttt{Retreive\ Data}. Take note when you check the
  \texttt{Unemployment\ Rate\ (Seasonally\ Adjusted)}, what is the
  series number that is associated with this?
\end{itemize}

What do you see when you get to the next page? A rectangular data series
that has months on the columns, years on the rows, and values as the
internals to the cells? :facepalm:

\begin{itemize}
\tightlist
\item
  Does this meet the requirements of
  \href{https://r4ds.had.co.nz/tidy-data.html}{tidy data}, or
  \href{https://tsibble.tidyverts.org}{time series tidy data}?
\item
  If you were to build an analytic pipeline against data that you
  accessed in this way, what would be the process to update your
  analysis when the next edition of data is released? Would it require a
  manual download, then cleaning, then movement into your analysis?
  Could this be problematic?
\end{itemize}

This motivates the idea of using the BLS' data API. The data API
provides consistently formatted JSON objects that can be converted to
data of an arbitrary (that is, useful to us) formatting. Because the
data is being provided in a JSON object, there is some work to coerce it
to be useful, but we'll find that there are so many people who are doing
this same coercion that there are ready-made wrappers that will help us
to do this work.

As an example, you can view how these JSON objects are formatted by
navigating to an API endpoint in your browser. Here is the endpoint for
the national unemployment:
{[}\href{https://api.bls.gov/publicAPI/v2/timeseries/data/LNS14000000}{link}{]}.

Let's pull unemployment from the BLS data API.

\begin{enumerate}
\def\labelenumi{\arabic{enumi}.}
\tightlist
\item
  Register for an API key with the BLS. You can register for this from
  the BLS' ``Getting Started'' page. They will then send you an API key
  to the email that you affiliate.
\item
  Find the series that we want to access. Frankly, this is part of
  accessing this API that is the most surprisingly difficult -- the BLS
  does not publish a list of the data series. From their
  \href{https://www.bls.gov/data/tools.htm}{Data Retrieval Tools} page
  there are links to popular series, a table lookup, and a Data Finder.
  \href{https://www.bls.gov/help/hlpforma.htm\#OEUS}{Elsewhere} they
  provide pages that describe how series IDs are formatted, but finding
  series still requires considerable meta-knowledge.
\end{enumerate}

For this assignment, consider the following three series:

\begin{enumerate}
\def\labelenumi{\arabic{enumi}.}
\tightlist
\item
  Total unemployment: \texttt{LNS14000000}
\item
  Male unemployment: \texttt{LNS14000001}
\item
  Female unemployment: \texttt{LNS14000002}
\end{enumerate}

Our goal is to analyze these three series for the last 20 years.

To articulate the BLS API, we have found the \texttt{blsR} library to be
the most effective (at the time that we wrote the assignment in 2022).
Here are links to get you read into the package. Rather than providing
you with a \emph{full} walk-through for how to use this package to
manipulate the BLS data API, instead a learning goal is for you to read
these documents and come to an understanding of how the package works.

\begin{itemize}
\tightlist
\item
  \href{https://cran.r-project.org/web/packages/blsR/index.html}{CRAN
  Homepage}
\item
  \href{https://github.com/groditi/blsR}{GitHub}
\item
  \href{https://cran.r-project.org/web/packages/blsR/readme/README.html}{Vignette}
  (Called, incorrectly a README on the CRAN page)
\end{itemize}

\subsection{(2 points) Form a successful query and tidy of
data}\label{points-form-a-successful-query-and-tidy-of-data}

Your task is to create an object called \texttt{unemployment} that is a
\texttt{tsibble} class, that contains the overall unemployment rate, as
well as the unemployment rate for male and female people.

Your target dataframe should have the following shape but extend to the
current time period.

\begin{verbatim}
year month time_index name    value
   <int> <int>      <mth> <chr>   <dbl>
 1  2000     1   2000 Jan overall   4  
 2  2000     1   2000 Jan male      3.9
 3  2000     1   2000 Jan female    4.1
 4  2000     2   2000 Feb overall   4.1
 5  2000     2   2000 Feb male      4.1
 6  2000     2   2000 Feb female    4.1
 7  2000     3   2000 Mar overall   4  
 8  2000     3   2000 Mar male      3.8
 9  2000     3   2000 Mar female    4.3
10  2000     4   2000 Apr overall   3.8
\end{verbatim}

\newpage

\subsubsection{Answer}\label{answer-8}

\paragraph{Query data}\label{query-data}

I defined current\_year programatically and saved the qurery of the 3
series in one df.

\paragraph{Tidy data}\label{tidy-data}

Manually downloaded data with year and month as rows and three series as
columns, i.e., in a wide format. Tidy data requires a long format, where
each observation (a monthly unemployment reading) is a single row, and
each variable (year, month, series\_id, value) is a separate column.

So I pivot the 3 seires (col) names into one col with 3 values/factors.

\paragraph{Using API as opposed to manually downloaded
data}\label{using-api-as-opposed-to-manually-downloaded-data}

Doing analysis on manually downloaded data requires a manual update
(download, cleaning, and integration) every time the BLS releases new
data. This process is non-reproducible, not scalable, and error-prone,
making the analytic pipeline difficult to maintain. Using the blsR API
package makes the process automated and programmatic.

\paragraph{Disclosure}\label{disclosure}

Code is produced with help from AI but I checked every line of code and
made many edits.

\begin{Shaded}
\begin{Highlighting}[]
\FunctionTok{Sys.setenv}\NormalTok{(}\AttributeTok{BLS\_KEY =} \StringTok{"https://doi.org/10.4324/9781315157375"}\NormalTok{)}

\NormalTok{series\_id\_list }\OtherTok{\textless{}{-}} \FunctionTok{c}\NormalTok{(}
  \AttributeTok{overall =} \StringTok{"LNS14000000"}\NormalTok{,}
  \AttributeTok{male    =} \StringTok{"LNS14000001"}\NormalTok{,}
  \AttributeTok{female  =} \StringTok{"LNS14000002"}
\NormalTok{)}

\NormalTok{current\_year }\OtherTok{\textless{}{-}} \FunctionTok{as.numeric}\NormalTok{(}\FunctionTok{format}\NormalTok{(}\FunctionTok{Sys.Date}\NormalTok{(), }\StringTok{"\%Y"}\NormalTok{))}
\NormalTok{current\_year}
\end{Highlighting}
\end{Shaded}

\begin{verbatim}
## [1] 2025
\end{verbatim}

\begin{Shaded}
\begin{Highlighting}[]
\CommentTok{\# Query the BLS API for the last 20 years (e.g., 2004 to current year)}
\CommentTok{\# Note: The output is a complex list structure}
\NormalTok{raw\_bls\_data }\OtherTok{\textless{}{-}}\NormalTok{ blsR}\SpecialCharTok{::}\FunctionTok{get\_n\_series\_table}\NormalTok{(}
\NormalTok{  series\_id\_list,}
  \CommentTok{\# Request 20 years of data}
  \AttributeTok{start\_year =}\NormalTok{ current\_year }\SpecialCharTok{{-}} \DecValTok{20}\NormalTok{,}
  \AttributeTok{end\_year =}\NormalTok{ current\_year}
\NormalTok{)}
\end{Highlighting}
\end{Shaded}

\begin{verbatim}
## Year 2005 to 2025 is longer than 10 year API limit. Performing 3 requests.
\end{verbatim}

\begin{verbatim}
## Warning: api_key is required for multiple series requests.
## api_key is required for multiple series requests.
## api_key is required for multiple series requests.
\end{verbatim}

\begin{Shaded}
\begin{Highlighting}[]
\FunctionTok{dim}\NormalTok{(raw\_bls\_data)}
\end{Highlighting}
\end{Shaded}

\begin{verbatim}
## [1] 248   5
\end{verbatim}

\begin{Shaded}
\begin{Highlighting}[]
\FunctionTok{class}\NormalTok{(raw\_bls\_data)}
\end{Highlighting}
\end{Shaded}

\begin{verbatim}
## [1] "tbl_df"     "tbl"        "data.frame"
\end{verbatim}

\begin{Shaded}
\begin{Highlighting}[]
\NormalTok{unemployment }\OtherTok{\textless{}{-}}\NormalTok{ raw\_bls\_data }\SpecialCharTok{\%\textgreater{}\%}
  \CommentTok{\# 1. Pivot the data from wide to long format}
  \FunctionTok{pivot\_longer}\NormalTok{(}
    \AttributeTok{cols =} \FunctionTok{starts\_with}\NormalTok{(}\StringTok{"LNS"}\NormalTok{), }\CommentTok{\# Select all columns starting with the Series ID prefix}
    \AttributeTok{names\_to =} \StringTok{"series\_id"}\NormalTok{,}
    \AttributeTok{values\_to =} \StringTok{"value"}
\NormalTok{  ) }\SpecialCharTok{\%\textgreater{}\%}
  
  \CommentTok{\# 2. Convert BLS period (e.g., "M01" {-}\textgreater{} "1") into a proper date object}
  \FunctionTok{mutate}\NormalTok{(}
    \CommentTok{\# Create the date by combining year and the period (month number)}
    \AttributeTok{date =}\NormalTok{ lubridate}\SpecialCharTok{::}\FunctionTok{ymd}\NormalTok{(}\FunctionTok{paste0}\NormalTok{(year, }\StringTok{"{-}"}\NormalTok{, }\FunctionTok{gsub}\NormalTok{(}\StringTok{"M"}\NormalTok{, }\StringTok{""}\NormalTok{, period), }\StringTok{"{-}01"}\NormalTok{)),}
    
    \CommentTok{\# Clean up the series IDs into readable names}
    \AttributeTok{name =} \FunctionTok{case\_when}\NormalTok{(}
\NormalTok{      series\_id }\SpecialCharTok{==} \StringTok{"LNS14000000"} \SpecialCharTok{\textasciitilde{}} \StringTok{"overall"}\NormalTok{,}
\NormalTok{      series\_id }\SpecialCharTok{==} \StringTok{"LNS14000001"} \SpecialCharTok{\textasciitilde{}} \StringTok{"male"}\NormalTok{,}
\NormalTok{      series\_id }\SpecialCharTok{==} \StringTok{"LNS14000002"} \SpecialCharTok{\textasciitilde{}} \StringTok{"female"}\NormalTok{,}
      \ConstantTok{TRUE} \SpecialCharTok{\textasciitilde{}}\NormalTok{ series\_id}
\NormalTok{    )}
\NormalTok{  ) }\SpecialCharTok{\%\textgreater{}\%}
  
  \CommentTok{\# 3. Finalize columns for tsibble casting}
  \FunctionTok{mutate}\NormalTok{(}
    \AttributeTok{year =}\NormalTok{ lubridate}\SpecialCharTok{::}\FunctionTok{year}\NormalTok{(date),}
    \AttributeTok{month =}\NormalTok{ lubridate}\SpecialCharTok{::}\FunctionTok{month}\NormalTok{(date),}
    \AttributeTok{time\_index =} \FunctionTok{yearmonth}\NormalTok{(date)}
\NormalTok{  ) }\SpecialCharTok{\%\textgreater{}\%}
  
  \CommentTok{\# 4. Cast to tsibble}
\NormalTok{  tsibble}\SpecialCharTok{::}\FunctionTok{as\_tsibble}\NormalTok{(}
    \AttributeTok{key =}\NormalTok{ name,}
    \AttributeTok{index =}\NormalTok{ time\_index}
\NormalTok{  ) }\SpecialCharTok{\%\textgreater{}\%}
  
  \CommentTok{\# 5. Select the required final columns and ensure value is numeric}
  \FunctionTok{select}\NormalTok{(year, month, time\_index, name, value) }\SpecialCharTok{\%\textgreater{}\%}
  \FunctionTok{mutate}\NormalTok{(}\AttributeTok{value =} \FunctionTok{as.numeric}\NormalTok{(value))}
\end{Highlighting}
\end{Shaded}

\begin{Shaded}
\begin{Highlighting}[]
\FunctionTok{head}\NormalTok{(unemployment)}
\end{Highlighting}
\end{Shaded}

\begin{verbatim}
## # A tsibble: 6 x 5 [1M]
## # Key:       name [1]
##    year month time_index name   value
##   <dbl> <dbl>      <mth> <chr>  <dbl>
## 1  2005     1   2005 Jan female   5.1
## 2  2005     2   2005 Feb female   5.3
## 3  2005     3   2005 Mar female   5.1
## 4  2005     4   2005 Apr female   5.2
## 5  2005     5   2005 May female   5.2
## 6  2005     6   2005 Jun female   5.1
\end{verbatim}

\newpage

\subsection{(1 point) Plot the Unemployment
Rate}\label{point-plot-the-unemployment-rate}

Once you have queried the data and have it successfully stored in an
appropriate object, produce a plot that shows the unemployment rate on
the y-axis, time on the x-axis, and each of the groups (overall, male,
and female) as a different colored line.

\subsubsection{Answer}\label{answer-9}

\textbf{Comment:} ACF Plot: The ACF coefficients start at 1 and show a
slow decay. This indicates non-stationarity and persistence:
unemployment levels change gradually. There is no significant peak at
Lag 12 (or multiples of 12), which may be because the data is seasonally
adjusted.

Lag Plot: The points form a mostly tight, upward-sloping cluster. This
visually confirms an high positive correlation between this month's
unemployment rate and next month's rate, showing that the series is
highly predictable from its immediate past. The few long lines indicates
the rare events where employment jumped due to exogenous shocks (Covid,
subprime financial crisis).

\begin{Shaded}
\begin{Highlighting}[]
\NormalTok{unemployment }\SpecialCharTok{\%\textgreater{}\%}
  \FunctionTok{ggplot}\NormalTok{(}\FunctionTok{aes}\NormalTok{(}\AttributeTok{x =}\NormalTok{ time\_index, }\AttributeTok{y =}\NormalTok{ value, }\AttributeTok{color =}\NormalTok{ name)) }\SpecialCharTok{+}
  \FunctionTok{geom\_line}\NormalTok{(}\AttributeTok{linewidth =} \DecValTok{1}\NormalTok{) }\SpecialCharTok{+} 
  \FunctionTok{labs}\NormalTok{(}\AttributeTok{title =} \StringTok{"Unemployment by gender: 2005 {-} 2025"}\NormalTok{,}
       \AttributeTok{x =} \StringTok{"time"}\NormalTok{)}
\end{Highlighting}
\end{Shaded}

\pandocbounded{\includegraphics[keepaspectratio]{unit_06_files/figure-latex/plot unemployment data-1.pdf}}

\subsection{(1 point) Plot the ACF and
Lags}\label{point-plot-the-acf-and-lags}

This should feel familiar by now: Produce the ACF and lag plot of the
\texttt{overall} unemployment series. What do you observe?

\begin{Shaded}
\begin{Highlighting}[]
\NormalTok{unemployment }\SpecialCharTok{\%\textgreater{}\%}
  \FunctionTok{filter}\NormalTok{(name }\SpecialCharTok{==} \StringTok{"overall"}\NormalTok{) }\SpecialCharTok{\%\textgreater{}\%}
  \FunctionTok{ACF}\NormalTok{(value, }\AttributeTok{lag\_max =} \DecValTok{240}\NormalTok{) }\SpecialCharTok{\%\textgreater{}\%}
  \FunctionTok{autoplot}\NormalTok{() }\SpecialCharTok{+} 
  \FunctionTok{labs}\NormalTok{(}\AttributeTok{title =} \StringTok{"Overall Unemployment Rate ACF"}\NormalTok{)}
\end{Highlighting}
\end{Shaded}

\pandocbounded{\includegraphics[keepaspectratio]{unit_06_files/figure-latex/trends plot-1.pdf}}

\begin{Shaded}
\begin{Highlighting}[]
\NormalTok{unemployment }\SpecialCharTok{\%\textgreater{}\%}
  \FunctionTok{filter}\NormalTok{(name }\SpecialCharTok{==} \StringTok{"overall"}\NormalTok{) }\SpecialCharTok{\%\textgreater{}\%}
  \FunctionTok{gg\_lag}\NormalTok{(value, }\AttributeTok{.lag =} \DecValTok{9}\NormalTok{, }\AttributeTok{color =} \DecValTok{1}
\NormalTok{         ) }\SpecialCharTok{+} 
  \FunctionTok{labs}\NormalTok{(}\AttributeTok{title =} \StringTok{"Overall Unemployment Rate Lag Plot"}\NormalTok{)}
\end{Highlighting}
\end{Shaded}

\begin{verbatim}
## Warning in lag_geom(..., arrow = arrow): Ignoring unknown parameters: `.lag`
\end{verbatim}

\pandocbounded{\includegraphics[keepaspectratio]{unit_06_files/figure-latex/trends plot-2.pdf}}

\end{document}
